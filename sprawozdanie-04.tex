\documentclass[10pt,a4paper]{article}
\usepackage[a4paper]{geometry}

\usepackage{polski}
\usepackage{xltxtra}
\usepackage{indentfirst}
\usepackage{relsize}
\usepackage{fancyvrb}
\usepackage[pdfborder={0 0 0}]{hyperref}
\usepackage{graphicx}
\usepackage{changepage}

%% tweak fonts
\defaultfontfeatures{Mapping=tex-text}
\setromanfont{Charis SIL}
\setsansfont[Scale=MatchLowercase]{Helvetica Neue}
\setmonofont[Scale=MatchLowercase]{Menlo}
\linespread{1.25}

%% define custom commands and environments
\DefineVerbatimEnvironment%
  {SmallVerbatim}%
  {Verbatim}{fontsize=\relsize{-0.5},numbers=left,numbersep=-10pt,frame=lines,tabsize=4}

\newcommand{\f}[1]{\texttt{#1}}
\newcommand{\s}[1]{\textsf{#1}}

\begin{document}

%%fakesection{Tytuł}
\title{
  Sprawozdanie nr~4 z~laboratorium\\Podstaw Inżynierii Oprogramowania
}
\author{
  Grzegorz Bartkowiak\\
  Tomasz Cudziło\\
  Mateusz Ochtera\\
  Gustaw Wypych\\
  \\
  \textsc{PW EE Informatyka}\\[10pt]
}
% \date{\today}
\date{23 maja 2011}
\maketitle

\section{Stany i~zachowania systemu}

System aukcyjny koordynuje stany przedmiotów wystawionych na aukcjach oraz
reaguje na akcje podejmowane przez pracowników i~klientów domu aukcyjnego.
Dokument zawiera diagramy dotyczące co bardziej skomplikowanych przypadków.

\subsection{Rejestracja użytkownika}

Rejestracja klienta w~systemie jest wymagana, by mógł on skorzystać
z~jakichkolwiek funkcji systemu. Schemat opisanych aktywności znajduje się na
str.~\pageref{fig:aktywnosc_rejestracja_uzytkownika}.

Obowiązkowe jest zaakceptowanie regulaminu. Po akceptacji, klient proszony jest
o wypełnienie formularza z~danymi personalnymi. W~przypadku braku akceptacji
proces rejestracji jest anulowany i kończony. Po uzupełnieniu formularza,
następuje weryfikacja wprowadzonych danych. W przypadku niespełnienia dowolnego
wymogu, pojawia się monit z~prośbą o poprawienie błędów.

Po pomyślnym przejściu weryfikacji, następuje weryfikacja adresu email gdzie
wysyłane są tymczasowe kody do aktywowania konta z ograniczeniami. Jednocześnie
wysyłane są kurierem kody na adres domowy użytkownika. W przypadku nie
wprowadzenia kodów dostarczonych przez kuriera konto jest aktywne ale ma
ograniczenia, natomiast po wprowadzeniu kodów następuje pełna aktywacja konta.

Rezygnacja jest możliwa do momentu weryfikacji adresu email. Po zweryfikowaniu,
rezygnacja skutkuje anulowaniem procesu rejestracji i zakończeniem całego
procesu rejestracji.

\newpage
\begin{figure}[p]
  \begin{adjustwidth}{-3cm}{-3cm}
    \centering
    \includegraphics{figury/aktywnosc-rejestracja-uzytkownika}
    \caption{Diagram aktywności procesu rejestracji użytkownika.}
    \label{fig:aktywnosc_rejestracja_uzytkownika}
  \end{adjustwidth}
\end{figure}
\clearpage

\subsection{Stany przedmiotu w~domu akcyjnym}

Głównym zadaniem systemu jest zarządzanie przedmiotami, które na czas trwania
aukcji są pod opieką domu aukcyjnego. Diagram obrazujący możliwe stany w~jakich
przedmiot aukcji może się znaleźć jest na stronie \pageref{fig:stan_przedmiot}.

Zanim właściciel wyśle przedmiot do magazynu domu aukcyjnego musi
wypełnić specjalny formularz (dostępny na stronie) i przesłać go razem z
paczką.

Kurier odbiera od klienta paczkę i składa ją w magazynie. Przed przyjęciem
zawartość paczki zostaje odpowiednio oznakowana i wprowadzona do systemu.

Towar zostaje w magazynie dopóki nie zostanie zakupiony na aukcji lub gdy
właściciel wycofa przedmiot. Po zakończonej aukcji paczka trafia do nabywcy, a
w serwisie powstaje nowy wpis na ten temat.

\newpage
\begin{figure}[p]
  \begin{adjustwidth}{-3cm}{-3cm}
    \centering
    \includegraphics{figury/stan-przedmiot}
    \caption{Diagram stanów dla cyklu życia przedmiotu w~systemie aukcyjnym.}
    \label{fig:stan_przedmiot}
  \end{adjustwidth}
\end{figure}
\clearpage

\subsection{Wystawienie przedmiotu}

Proces wystawiania przedmiotu na aukcji jest możliwie jak najprostszy, ze
względu na wygodę użytkownika. Całość jest obsługiwana automatycznie przez
system, wymagając jak najmniej interakcji ze strony klienta.

W~przypadku gdy klient nie dostarczył do domu aukcyjnego żadnych przedmiotów,
nie może oczywiście rozpocząć aukcji.

\vspace{40pt}
\begin{figure}[hb]
  \begin{adjustwidth}{-3cm}{-3cm}
    \centering
    \includegraphics{figury/aktywnosc-nowa-aukcja}
    \caption{Diagram stanów dla cyklu życia przedmiotu w~systemie aukcyjnym.}
    \label{fig:aktywnosc_nowa_aukcja}
  \end{adjustwidth}
\end{figure}

\end{document}

